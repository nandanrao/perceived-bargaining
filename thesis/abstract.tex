\documentclass[a4paper,12pt]{article}
\usepackage{ fancyhdr, pgfplots}
\usepackage{fancyvrb}
\usepackage[backend=biber, citestyle=authoryear]{biblatex}
\renewcommand*{\nameyeardelim}{\addcomma\space}
\addbibresource{thesis.bib}

\pagestyle{fancy}
\fancyhf{}
\rhead{}
\lhead{ Numerocity Biases and the Perceived Chances of Getting a Job }
\rfoot{\thepage}

\begin{document}

\title{ Numerocity Biases and the Perceived Chances of Getting a Job: Experimental Evidence and Implications for Directed Search }

\author{Nandan Rao}

\maketitle

\begin{abstract}

In the labor market, a clear implication of internet-based job applications is the ability of every worker to apply to more jobs without a corresponding increase in the firms' capacity to interview more workers. In this paper, I consider the interaction between the congestion caused by this imbalance and the search behavior of workers.

I hypothesize, and show preliminary experimental evidence that, statistical biases related to numerocity can affect probabilistic judgement in such a way as to cause misjudgements: faster ``applying'' is shown to potentially lead to suboptimal early ``quitting.''

I use estimations from the experimental evidence to calibrate a slightly extended version of the equilibrium directed search model of \cite{gonzalez2010}. Simulations show that this bias causes a hollowing-out of the middle class from the wage distribution, as high-skill workers who experience rejection direct their search to jobs with lower-than-optimal wages.

\end{abstract}

\end{document}